\documentclass[sigconf,screen]{acmart}

\settopmatter{printacmref=false} % Removes citation information below abstract
\renewcommand\footnotetextcopyrightpermission[1]{} % removes footnote with conference information in first column
\pagestyle{plain} % removes running headers


%%
%% \BibTeX command to typeset BibTeX logo in the docs
\AtBeginDocument{%
  \providecommand\BibTeX{{%
    \normalfont B\kern-0.5em{\scshape i\kern-0.25em b}\kern-0.8em\TeX}}}


%% These commands are for a PROCEEDINGS abstract or paper.
\acmConference[Data Science for Software Engineering]{Data Science for Software Engineering}{Fall 2020}{Zurich, CH}


%%
%% end of the preamble, start of the body of the document source.
\begin{document}

%%
%% The "title" command has an optional parameter,
%% allowing the author to define a "short title" to be used in page headers.
\title{Reproduction of "Predicting the Severity of a Reported Bug"}

%%
%% The "author" command and its associated commands are used to define
%% the authors and their affiliations.
%% Of note is the shared affiliation of the first two authors, and the
%% "authornote" and "authornotemark" commands
%% used to denote shared contribution to the research.
\author{Andrina Vincenz}
\email{andrina.vincenz@uzh.ch}
\affiliation{%
  \institution{University of Zurich}
  \city{Zurich}
  \country{Switzerland}
}

\author{David Stalder}
\email{david.stalder@uzh.ch}
\affiliation{%
  \institution{University of Zurich}
  \city{Zurich}
  \country{Switzerland}
}


%%
%% By default, the full list of authors will be used in the page
%% headers. Often, this list is too long, and will overlap
%% other information printed in the page headers. This command allows
%% the author to define a more concise list
%% of authors' names for this purpose.
\renewcommand{\shortauthors}{Trovato, Tobin, B\'eranger}

%%
%% The abstract is a short summary of the work to be presented in the
%% article.
\begin{abstract}

\textbf{Background.} Bug fixing is a fundamental part of software maintenance. Bugs have different levels of severity according to their threat to the system and urgency to fixing.

\textbf{Aim.} The aim is to be able to have models which are able to identify the severity of bug reports.

\textbf{Method.} After identifying the system, bug reports will be extracted from bugzilla including the severity of the bug. This data is used to train two classifiers: SGDClassifier and Multinomial Na{\"\i}ve Bayes. The following measurements are used to identify the quality of the model: precision, recall and f-measures.

\textbf{Conclusion.} This study will show if it is possible to generate models, which can do the classification of bug reports automatically with certainty. This can help software engineers in practice to classify their bug reports and focus on the most urgent.
\end{abstract}


%%
%% This command processes the author and affiliation and title
%% information and builds the first part of the formatted document.
\maketitle

\section{Introduction}
Bug reports represent a fundamental part of software maintenance by documenting the malfunction of a software part. A large number of bug reports by users and developers are collected on a daily basis in bug tracking systems like Bugzilla. These reports are then manually prioritized according to their severity and classified into 6 groups, ranging from a request to enhancement to a critical level resulting in crashes, loss of data or severe memory leak. The level of severity indicates the impact of the bug on the successful execution of a software system.

However, the manual assignment of the severity level to a bug report takes time and resources. In combination with the increase of daily bug reports the classification introduces a set of challenges such as misclassification of severity or the increase of cost. The ability of the automated severity prediction of bug reports by a tool reduces errors in misclassification and manual work as well as enhance the software quality.

In order to automate the severity predication of bug reports, different classification and information retrieval techniques based on textual descriptions have been used previously. Techniques range from Nearest Neighbours \cite{TianDrone} \cite{LamkanfiMiningAlgo} \cite{ChaturvediBugSeverity}, Na{\"\i}ve Bayes \cite{LamkanfiMiningAlgo} \cite{ChaturvediBugSeverity} to Support Vector Machines \cite{KanwalOMSVM} \cite{LamkanfiMiningAlgo} \cite{ChaturvediBugSeverity}. While some work focuses on automatically analyzing the text of past bug reports and the recommendation of severity labels \cite{TianDrone} \cite{LamkanfiMiningAlgo} \cite{ChaturvediBugSeverity} other work creates a classification-based approach to create a bug priority recommender \cite{KanwalOMSVM}.

For this project, we intend to conduct a reproduction of the paper "Predicting the severity of a reported bug" \cite{ourPaper} by Lamkanfy et al. It was originally written in 2010, which we consider as too old for such a current topic. 


%%
%% The next two lines define the bibliography style to be used, and
%% the bibliography file.
\bibliographystyle{ACM-Reference-Format}
\bibliography{sample-base}

%%
%% If your work has an appendix, this is the place to put it.
\appendix




\end{document}
\endinput
%%
%% End of file `sample-sigconf.tex'.
